%%%%%%%%%%%%%%%%%
% This is an sample CV template created using altacv.cls
% (v1.6, 21 May 2021) written by LianTze Lim (liantze@gmail.com). Now compiles with pdfLaTeX, XeLaTeX and LuaLaTeX.
%
%% It may be distributed and/or modified under the
%% conditions of the LaTeX Project Public License, either version 1.3
%% of this license or (at your option) any later version.
%% The latest version of this license is in
%%    http://www.latex-project.org/lppl.txt
%% and version 1.3 or later is part of all distributions of LaTeX
%% version 2003/12/01 or later.
%%%%%%%%%%%%%%%%

%% Use the "normalphoto" option if you want a normal photo instead of cropped to a circle
% \documentclass[10pt,a4paper,normalphoto]{altacv}

\documentclass[10pt,a4paper,ragged2e,withhyper]{altacv}
%% AltaCV uses the fontawesome5 and packages.
%% See http://texdoc.net/pkg/fontawesome5 for full list of symbols.

% Change the page layout if you need to
\geometry{left=1.25cm,right=1.25cm,top=1.5cm,bottom=1.5cm,columnsep=1.2cm}

% The paracol package lets you typeset columns of text in parallel
\usepackage{paracol}

% Change the font if you want to, depending on whether
% you're using pdflatex or xelatex/lualatex
\ifxetexorluatex
  % If using xelatex or lualatex:
  \setmainfont{Roboto Slab}
  \setsansfont{Lato}
  \renewcommand{\familydefault}{\sfdefault}
\else
  % If using pdflatex:
  \usepackage[rm]{roboto}
  \usepackage[defaultsans]{lato}
  % \usepackage{sourcesanspro}
  \renewcommand{\familydefault}{\sfdefault}
\fi

% Change the colours if you want to
\definecolor{SlateGrey}{HTML}{2E2E2E}
\definecolor{LightGrey}{HTML}{666666}
\definecolor{DarkPastelRed}{HTML}{450808}
\definecolor{PastelRed}{HTML}{8F0D0D}
\definecolor{GoldenEarth}{HTML}{E7D192}
\colorlet{name}{black}
\colorlet{tagline}{PastelRed}
\colorlet{heading}{DarkPastelRed}
\colorlet{headingrule}{GoldenEarth}
\colorlet{subheading}{PastelRed}
\colorlet{accent}{PastelRed}
\colorlet{emphasis}{SlateGrey}
\colorlet{body}{LightGrey}

% Change some fonts, if necessary
\renewcommand{\namefont}{\Huge\rmfamily\bfseries}
\renewcommand{\personalinfofont}{\footnotesize}
\renewcommand{\cvsectionfont}{\LARGE\rmfamily\bfseries}
\renewcommand{\cvsubsectionfont}{\large\bfseries}


% Change the bullets for itemize and rating marker
% for \cvskill if you want to
\renewcommand{\itemmarker}{{\small\textbullet}}
\renewcommand{\ratingmarker}{\faCircle}

\begin{document}
\name{William Dixon}
\tagline{Senior Site Reliability Engineer}
%% You can add multiple photos on the left or right
%%\photoR{2.8cm}{Self_Port}
% \photoL{2.5cm}{Yacht_High,Suitcase_High}

\personalinfo{%
  % Not all of these are required!
  \email{will@willd.io}
  \email{dixonwille@gmail.com}
  \phone{(843)617-6097}
  \mailaddress{1934 Horlbeck St. Florence SC, 29505}
  \linkedin{dixonwille}
  \github{dixonwille}
  %% You can add your own arbitrary detail with
  %% \printinfo{symbol}{detail}[optional hyperlink prefix]
  % \printinfo{\faPaw}{Hey ho!}[https://example.com/]
  %% Or you can declare your own field with
  %% \NewInfoFiled{fieldname}{symbol}[optional hyperlink prefix] and use it:
  % \NewInfoField{gitlab}{\faGitlab}[https://gitlab.com/]
  % \gitlab{your_id}
  %%
  %% For services and platforms like Mastodon where there isn't a
  %% straightforward relation between the user ID/nickname and the hyperlink,
  %% you can use \printinfo directly e.g.
  % \printinfo{\faMastodon}{@username@instace}[https://instance.url/@username]
  %% But if you absolutely want to create new dedicated info fields for
  %% such platforms, then use \NewInfoField* with a star:
  % \NewInfoField*{mastodon}{\faMastodon}
  %% then you can use \mastodon, with TWO arguments where the 2nd argument is
  %% the full hyperlink.
  % \mastodon{@username@instance}{https://instance.url/@username}
}

\makecvheader
%% Depending on your tastes, you may want to make fonts of itemize environments slightly smaller
% \AtBeginEnvironment{itemize}{\small}

%% Set the left/right column width ratio to 6:4.
\columnratio{0.72}

% Start a 2-column paracol. Both the left and right columns will automatically
% break across pages if things get too long.
\begin{paracol}{2}
\cvsection{Experience}

\cvevent{ACS Technologies}{Site Reliability Engineer}{October 2018 -- Present}{}
The main goal of the SREs is to maintain and improve the reliability of our systems. We had an on call
schedule so individuals could report an incedent and an SRE would triage and resolve the issue in a
timely fashion. Tools were created to increase the reliability of our systems by reducing the human
factor. Major leaps were made with our microservice platform, our pipelines, and the infrastructure
our services were on. To top it all off, we used Datadog to activly monitor our systems and services.

\cvevent{}{Software Engineer}{September 2015 -- October 2018}{}
Software Engineer was focused on creating quality features. Using C\#'s NUnit framework and Go's built-in
testing package, we were able to get some unit and integration tests in place. It was expected that a software
engineer understood an API request's flow through our software to know when it was best to call SQL or
create an asyncronous job instead. We had to understand the basics of both frontend and backend as it was
expected for an engineer to work in both areas, though it was expected one would be stronger in one or the other.
I was strongest in the backend but did my fair share in the frontend when needed.

\cvsection{Projects}

\cvevent{Kubernetes Migration}{}{}{}
ACS's infrastructure was built on AWS using AWS stack. In efforts to reduce the complexity of creating our AWS
version of a container orchestrator, we worked on getting our containers deployed to a Kubernetes cluster instead
of individual EC2 instances. We couldn't remove any features that existed already (metrics, monitors, etc). We used Helm,
a few open source Kubernetes Operators, and even wrote our own operator to handle some of the more complex portions
of our deployments.

\cvevent{Pipelines}{}{}{}
To increase time to production of a feature, it is key to have a pipeline in place that can put quality gates in place
so we can be confident in the features being produced. We started off using Jenkins and started moving to Github Actions.
This pipeline had to be aware of exactly what was changed inside of our monolithic repository as to not build things that
were not affected by a change. A custom tool for Go was built that would return a list of Go packages that were
affected with a specific git diff (immediate and dependents). The pipeline was built to build, lint, run unit test, and create
artifacts that were deployable. A separate pipeline was created that consumed those artifacts that were environment agnostic
and combine them with configs on deployment of the service.

\cvevent{Monolithic Repository and Microservice Platform}{}{}{}
We wanted to introduce microservices, but didn't want to introduce the headache of many git repositories to maintain.
So we decided to create a monolithic repository for our microservices. We also needed to build a microservice platform
as the existing platforms did not meet the requirements needed by some of our legacy code bases. We wrote a protocol
buffer transport platform influenced by GRPC but had the flexability to also allowed JSON for our legacy code bases without
a proxy in front of the service.

\medskip

%\cvsection{A Day of My Life}

% Adapted from @Jake's answer from http://tex.stackexchange.com/a/82729/226
% \wheelchart{outer radius}{inner radius}{
% comma-separated list of value/text width/color/detail}
%\wheelchart{1.5cm}{0.5cm}{%
%  6/8em/accent!30/{Sleep,\\beautiful sleep},
%  3/8em/accent!40/Hopeful novelist by night,
%  8/8em/accent!60/Daytime job,
%  2/10em/accent/Sports and relaxation,
%  5/6em/accent!20/Spending time with family
%}

% use ONLY \newpage if you want to force a page break for
% ONLY the current column
%\newpage


%% Switch to the right column. This will now automatically move to the second
%% page if the content is too long.
\switchcolumn

\cvsection{Philosophy}

\begin{quote}
"If you can't explain it\\
simply, you don't\\
understand it well enough."
\end{quote}

-- Albert Einstein

\cvsection{Programming}

\cvskill{Go}{5}
\cvskill{Rust}{4}
\cvskill{C\#}{4}
\cvskill{Bash}{3.5}

\divider \smallskip

\cvtag{HTML}
\cvtag{CSS}
\cvtag{Javascript}\\
\cvtag{Java}
\cvtag{SQL}
\cvtag{C}

\cvsection{Technologies}

\cvskill{Docker}{5}
\cvskill{Pipelines}{4.5}
\cvskill{Kubernetes}{4}
\cvskill{AWS}{4}

\divider \smallskip

\cvtag{Data Structures}\\
\cvtag{Microservice Arch}
\cvtag{NoSQL}
\cvtag{Datadog}

\cvsection{Interests}

\cvtag{Embedded Development}\\
\cvtag{Automation}
\cvtag{Security}\\
\cvtag{Reliability}

\cvsection{Office}

\cvtag{Google Suite}
\cvtag{Jira}\\
\cvtag{Confluence}
\cvtag{Slack}
\cvtag{Scrum}
\cvtag{Kanban}

\cvsection{Referees}

% \cvref{name}{email}{mailing address}
\cvref{Joshua Suggs}{ACS Technologies}{josh@acst.com}

\end{paracol}


\end{document}